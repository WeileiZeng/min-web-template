\def\rank{\mathop{\rm rank}}
\def\wgt{\mathop{\rm wgt}}
\def\sgn{\mathop{\rm sgn}}
\def\lc{\mathop{\rm lc}}
\documentclass[aps,prb,12pt,tightenlines,%
notitlepage,longbibliography
]{revtex4-1}

\usepackage{xcolor}
\usepackage{hyperref}
\usepackage{graphicx}
\graphicspath{{../figure/gnuplot/surf/}}
\usepackage{amsthm}
\usepackage{amsfonts}
\usepackage{array}
\usepackage{enumitem}
\newtheorem{theorem}{Theorem}
\newtheorem{statement}[theorem]{Theorem}
\newtheorem{note}[theorem]{Note}
\newtheorem{corollary}[theorem]{Corollary}
\newtheorem{conjecture}[theorem]{Conjecture}
\newtheorem{lemma}[theorem]{Lemma}
\newtheorem{definition}[theorem]{Definition}
%\advance\textheight by -3.2in
\begin{document}
\title{Note on Belief Propagation}
\author{Weilei Zeng}
\date{\today}
\begin{abstract}
  This work gives a detail description of Belief Propagation (BP)
  algorithm and several of its variations.
\end{abstract}

\maketitle


\section{description of binary BP decoder}
The Belief Propagation (BP) algorithm can approximate the maginal
probability of errors, giving the received bits or the syndrome. It is
designed for factor graph. When the graph is a tree, it gives the
exact marginal probability. There are several equivalent/approximate
versions of it, let us start with the classical one.

\subsection{codeword-based}
BP decoder for classical binary codes:

A code is defined by a factor graph ( with assumed degree $d+1$ and) with variable nodes $V=\{c_i\}$
and check nodes $C=\{c_i\}$. The message from variable to check is
$$u_{v-c}(x) =m(x) = P(y_v|x) \prod_{j=1..d}  M_j(x)  \text {, where }
\{c_1,...,c_d \} = N(v)\setminus c,~x \in \{0,~1\}$$
The message from check to variable is
$$u_{c-v} = M(x) = \sum_{x_1,...,x_d} \delta( {\bf h}_c {\bf x}^T=0)
\prod_{j=1}^d m_j(x) \text{, where} \{ v_1,...,v_d \} = N(c)
\setminus v$$
The initial condition is $M_j(x)=1, x \in \{0,1\}$.

(To generalize it to nonbinary case, one only need to modify the
parity check condition from matrix miltiplication to symplectic
product. Everything else will be extended smoothly)

\subsection{codeword-based, LLR-simplified}
By rewriting the probabilities and messages in terms of the
log-likelihood ratio

$l_i=\log( m_i(0)/m_i(1))$, $M_i=\log(
M_i(0)/M_i(1))$, one can get the simplified message:

$$l_i=l_i^{(0)}+\sum_{j=1}^d (L_j)$$

$$L_i = 2 \tanh^{-1} \prod_{j=1}^d \tanh (l_j/2)$$

$$l^{(0)}_i=\log (P(x_i=0|y_i)/P(x_i=1)|y_i),~ y_i \in \{0,1\}$$,

When $y_i= 0 \text{ or } 1$, $l_i^{(0)}$ will be flipped.

In this simplified form, only one message need to be sent per edge,
instead of two messages for 0 and 1 respectively. Hence, the
complexity is reduced.

\subsection{syndrome-based}
In classical case, the received bits was used, instead of the
syndrome. In quantum case, there are no received bits, but only the
syndrome.
Hence, to get the formula in quantum case, one need to
change it to a syndrome-based decoder.

Ref \cite[chapter~47.2]{mackay2003information} show that the
codeword-based BP decoder is equivalent to the following
syndrome-based decoder

$$u_{v-c}(x) =m(x) = P(x) \prod_{j=1..d}  M_j(x)  \text {, where }
\{c_1,...,c_d \} = N(v)\setminus c$$

$$u_{c-v} = M(x) = \sum_{x_1,...,x_d} \delta( {\bf h}_c {\bf x}^T={\bf
s}^T)
\prod_{j=1}^d m_j(x) \text{, where} \{ v_1,...,v_d \} = N(c)
\setminus v$$
The initial condition is still $M_j(x)=1, x \in \{0,1\}$.

The logic is that, the codeword-view calculate $P(x|y)$ (the
most-likely input codeword $x$ given the recieved vector $y$) and the
syndrome-view calculate $P(e|s)$ (the most-likely error $e$ given this
syndrome $s$). Here $x$ should be an valid zero-syndrome
input codeword; $y$ is the received vector; and $e$ is an error vector
matching syndrome $s$. Literally, these two marginal probability are
describing the same event, thus should lead to the same
result. Mathematically, one has to write it carefully and show they
are isomorphic.

\subsection{syndrome-based, LLR simplified}
In a similar fashion of simplification, one can write the above
equations into the log-likelihood-ratio form, then reach the following
simple formula \cite{liu2018neural}

$$l_i=l_i^{(0)}+\sum_{j=1}^d (L_j)$$

$$L_i = (-1)^{s_i} 2 \tanh^{-1} \prod_{j=1}^d \tanh (l_j/2)$$

$$l_i^{(0)}=\log (P(x_i=0)/P(x_i=1)) = \text{const}$$

The posterior log-likelihood ratio can be estimated as
$l_i=l_i^{(0)}+\sum_{j=1}^{d+1} (L_j)$

This syndrome-based BP decoder can be used for quantum code as
well. We first discuss the case of CSS codes, then the case of GF(4)
codes.

\subsection{discussion on quantum case}

In CSS codes, one has $GH^T=0$. Say $H$ is the parity check matrix,
then the only difference from a classical code with parity check
matrix $H$ is that one need to check the decoded vector is an trivial
error or not, that is, if it can be eliminated
by rows of $G$ or not. Hence, the CSS code can use BP decoder directly
with a post check.

In GF(4) code, the generator matrix $G=(A|B)$ satisfies $G \tilde
G^T=AB^T+BA^T=0$, where $\tilde G=(B|A)$. Here, one can just decode a
classical code with parity check matrix $\tilde G$, then check if it
is a combination of rows of $G$.

Note that, in both CSS codes and GF(4) codes, the correlation between
X and Z errors are not considered. One way to consider the
correlations is as following.


One can change the
generator matrix from
$$G=(A|B), H=(B|A)$$
to
$$\tilde G =G \left( \begin{array}{ccc} I&&I\\&I&I \end{array}\right)
=(A|B|A+B), \tilde H = \left( \begin{array}{ccc} B & A \\
                                I&I&I \end{array} \right) $$
The error changes from $(e_X|e_Z)$ to $(e_X|e_Z|e_Y)$, which satisfy 
$(I|I|I)(e_X|e_Z|e_Y)^T=0 $. (Any single error will produce an
even number of 1s in the vector. This $e_Y$ is not the Pauli Y error,
but simply a superposition of X and Z, $e_Y=e_X+e_Z$ mod 2.) This
extra Y node contain the information that X and Z errors tend to
appear or disapear in pair but not alone.

In this construction, we can just take the new parity check matrix as
a classical binary code and use the basic BP decoder.


\subsection{variation of BP}
\subsubsection{Min-Sum}
Finally, there are some optimization of BP decoder, including
normalized and offset min-sum
decoder \cite{chen2005improved}. Ref~\cite{panteleev2019degenerate} says they are always using
normalized offset min-sum decoder with mormalization factor $\alpha=0.625$.

Here I use $L^{BP},~L^{MS},~L^{NORM},~L^{OFF}$ to denote the
check-to-variable messages
for BP, min-sum,
normalized min-sum, and off-set min-sum respectively. The relation on
their sign and magnitute are

$$\sgn(L^{BP})=\sgn(L^{MS})=(-1)^{s_c}\prod_i^d \sgn(l_j)$$

$$|L_i^{BP}| =  2 \tanh^{-1} \prod_{j=1}^d \tanh (|l_j|/2)$$

$$|L_i^{MS}| = \min_i^d |l_j|$$

$$|L_i^{NORM}| = \min_i^d |l_j|/\alpha,~\alpha>1$$

$$|L_i^{OFF}| = \max( \min_i^d |l_j| - \beta, 0),~\beta >0$$


\subsubsection{layered scheduling for updating rule}
Ref \cite{panteleev2019degenerate} claim they used layered scheduling,
which helped to eliminate the oscillating errors caused by the
trapping sets.
The criteria for how to choose the schedule is unclear for me yet.


\subsubsection{enhanced feedback}
Ref \cite{wang2012enhanced} developed an optimization called Enhanced
Feedback iterative BP decoder. In the second round of BP decoding, he
locate the frustrated checks and some common qubits connected with
them, then use the output probability to replace the input probability
for those qubits. This approach is very similar to what I tried ( in
the codeword-based LLR-simplified BP decoder for toric codes). The
difference is that, I simply use the output probability (LLR vector)
to replace the input probability for all qubits. I saw it fix all double errors
on large-size (about 35x35) toric codes, but only tiny improvement in
the numerics of small size (5, 7, 9, 11, 13).  I am not sure about the
reason on small size. there may be a bug in the program as well.


\iffalse
\begin{figure}[t]
\includegraphics[width=\linewidth]{../figure/gnuplot/iteration/rate-bp4-iteration20-compare0-cycle3000-logscale}

\includegraphics[width=\linewidth]{../figure/gnuplot/iteration/rate-bp4-iteration20-compare0-cycle3000-normal}

\caption{numerics on modified enhanced feedback}
\end{figure}
\fi
\bibliographystyle{alpha} %comment to show only numbers

\bibliography{WeileiBibFile}

\end{document}


